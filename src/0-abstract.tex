%%%%%
\begin{abstract}
Current task-granularity data dependency models require that all of the inputs to a task be computed before its execution begins. This paper describes the Fluid computation framework and language extensions that will enable programmers to easily express and manage the eager execution of consumer tasks in producer-consumer relationships via Fluid variables that provide controlled exposure of intermediate variable values based on introspective triggers relating to either or both of i) progress among producers and ii) variable value stability. A Fluid variable represents a variable on which a consumer function can operate even before it gets its final value (generated by a producer function).  Specifically, in this paper, we present the details of the programming language, compiler and runtime system support needed to realize Fluid computation, and report experimental results when applying Fluid computation to four different application programs that are amenable to approximation. Our experimental analysis indicates that the fluidized versions of  K-means, Bellman-Ford, graph coloring, and edge detection bring, on average, 59\%, 46\%, 16\% and 19\% execution time improvements, respectively, over their original counterparts, under the default values of our fluidization parameters.
\end{abstract}

%%%%%

%: First, where the input has already attained its final value; second, where the current value is indistinguishable from the final value from the perspective of the consumer; and third, where the current value, while distinguishable from the perspective of the consumer, has an effect on the output of the consumer that is within its fundamental tolerance for approximation.
